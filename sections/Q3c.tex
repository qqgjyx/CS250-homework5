\textbf{Solution:}

\noindent
Let's perform an initial simplification:

\textbf{For D0:}
\begin{align*}
    D0 &= !Q0 \cdot !Q1 \cdot !Q2 \cdot !IL1 \cdot IL0 + \\
       &\quad !Q0 \cdot Q1 \cdot Q2 \cdot IL1 \cdot !IL0 + \\
       &\quad Q0 \cdot !Q1 \cdot !Q2 \cdot !IL1 \cdot IL0 + \\
       &\quad Q0 \cdot Q1 \cdot Q2 \cdot IL1 \cdot !IL0 \\
       &= 
\end{align*}

\noindent
Using Boolean algebra, we can simplify the expression as follows (\cite{boolean-algebra}):
\begin{align*}
    D0 &= !Q0 \cdot !Q1 \cdot !Q2 \cdot !IL1 \cdot IL0 + \\
       &\quad !Q0 \cdot Q1 \cdot Q2 \cdot IL1 \cdot !IL0 + \\
       &\quad Q0 \cdot !Q1 \cdot !Q2 \cdot !IL1 \cdot IL0 + \\
       &\quad Q0 \cdot Q1 \cdot Q2 \cdot IL1 \cdot !IL0 \\
\end{align*}


\textbf{For D1:}
\begin{align*}
    D1 &= !Q0 \cdot !Q1 \cdot !Q2 \cdot !IL1 \cdot IL0 + \\
       &\quad !Q0 \cdot !Q1 \cdot !Q2 \cdot IL1 \cdot !IL0 \cdot IS + \\
       &\quad !Q0 \cdot !Q1 \cdot Q2 \cdot !IL1 \cdot IL0 \cdot IS + \\
       &\quad !Q0 \cdot !Q1 \cdot Q2 \cdot IL1 \cdot !IL0 \cdot !IS + \\
       &\quad !Q0 \cdot !Q1 \cdot Q2 \cdot IL1 \cdot !IL0 \cdot IS + \\
       &\quad !Q0 \cdot Q1 \cdot !Q2 \cdot !IL1 \cdot !IL0 + \\
       &\quad !Q0 \cdot Q1 \cdot !Q2 \cdot IL1 \cdot !IL0 \cdot !IS + \\
       &\quad !Q0 \cdot Q1 \cdot Q2 \cdot !IL1 \cdot !IL0+ \\
       &\quad !Q0 \cdot Q1 \cdot Q2 \cdot !IL1 \cdot IL0 \cdot IS + \\
       &\quad Q0 \cdot !Q1 \cdot !Q2 \cdot !IL1 \cdot !IL0 + \\
       &\quad Q0 \cdot !Q1 \cdot !Q2 \cdot !IL1 \cdot IL0 + IS \\
       &\quad Q0 \cdot Q1 \cdot Q2 \cdot !IL1 \cdot IL0 + \\
       &\quad Q0 \cdot Q1 \cdot Q2 \cdot IL1 \cdot !IL0 \cdot IS \\
    &= 
\end{align*}

\textbf{For D2:}
\begin{align*}
    D2 &= !Q0 \cdot !Q1 \cdot !Q2 \cdot !IL1 \cdot IL0 + \\
       &\quad !Q0 \cdot !Q1 \cdot !Q2 \cdot IL1 \cdot !IL0 \cdot !IS + \\
       &\quad !Q0 \cdot !Q1 \cdot Q2 \cdot !IL1 \cdot !IL0 + \\
       &\quad !Q0 \cdot !Q1 \cdot Q2 \cdot !IL1 \cdot IL0 \cdot IS + \\
       &\quad !Q0 \cdot !Q1 \cdot Q2 \cdot IL1 \cdot !IL0 \cdot IS + \\
       &\quad !Q0 \cdot Q1 \cdot !Q2 \cdot !IL1 \cdot IL0 \cdot !IS + \\
       &\quad !Q0 \cdot Q1 \cdot !Q2 \cdot IL1 \cdot !IL0 \cdot !IS + \\
       &\quad !Q0 \cdot Q1 \cdot Q2 \cdot !IL1 \cdot !IL0 + \\
       &\quad !Q0 \cdot Q1 \cdot Q2 \cdot !IL1 \cdot IL0 \cdot IS + \\
       &\quad Q0 \cdot !Q1 \cdot !Q2 \cdot !IL1 \cdot !IL0 + \\
       &\quad Q0 \cdot !Q1 \cdot !Q2 \cdot !IL1 \cdot IL0 \cdot IS + \\
       &\quad Q0 \cdot Q1 \cdot Q2 \cdot !IL1 \cdot IL0 + \\
       &\quad Q0 \cdot Q1 \cdot Q2 \cdot IL1 \cdot !IL0 \cdot !IS \\
    &= 
\end{align*}

\noindent
The logic expression of output bits are obivious:

\begin{align*}
    \text{out\_blocked} &=  Q0 \\
    \text{out\_posision1} &= !Q0 \cdot Q2 + Q0 \cdot !Q2 \\
    \text{out\_posision0} &= !Q0 \cdot Q1 + Q0 \cdot !Q1
\end{align*}

% ----------------------------------------------------------------
% Drafted by Juntang Wang 2024-10-22
% ----------------------------------------------------------------  
