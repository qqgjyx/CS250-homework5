\textbf{Solution:}

\textbf{Step 1: Identify the Instruction Format}

The instruction is \texttt{sw \$9, 4(\$2)}. This is a \emph{store word} instruction, which is an \textbf{I-type} instruction in MIPS. The I-type instruction format is:
\[
\text{[opcode (6 bits)]\quad[rs (5 bits)]\quad[rt (5 bits)]\quad[immediate (16 bits)]}
\]

\textbf{Step 2: Determine the Opcode and Register Numbers}

\begin{itemize}
    \item \textbf{Opcode}: The opcode for \texttt{sw} is \(43\) in decimal, which is \(101011\) in binary.\cite{MIPS32}
    \item \textbf{rs} (\emph{base register}): \(\$2\) corresponds to register number \(2\). In binary: \(00010\).
    \item \textbf{rt} (\emph{source register}): \(\$9\) corresponds to register number \(9\). In binary: \(01001\).
    \item \textbf{Immediate}: The offset is \(4\). In 16-bit binary: \(0000\ 0000\ 0000\ 0100\).
\end{itemize}

\textbf{Step 3: Assemble the Binary Instruction}

Concatenating these fields, we get the 32-bit instruction:
\[
\begin{array}{rl}
\text{Instruction Binary}: & 1\ 0\ 1\ 0\ 1\ 1\ 0\ 0\ 0\ 1\ 0\ 0\ 1\ 0\ 0\ 1\ 0\ 0\ 0\ 0\ 0\ 0\ 0\ 0\ 0\ 0\ 0\ 1\ 0\ 0 \\
\end{array}
\]

\textbf{Step 4: Group the Bits and Convert to Hexadecimal}

We will group the 32 bits into 8 groups of 4 bits, starting from bit 31 down to bit 0:
\[
\begin{array}{llcl}
\text{Group} & \text{Bits (from bit positions)} & \text{Binary} & \text{Hex} \\
\hline
\text{1} & \text{Bits 31--28} & 1\ 0\ 1\ 0 & \texttt{A} \\
\text{2} & \text{Bits 27--24} & 1\ 1\ 0\ 0 & \texttt{C} \\
\text{3} & \text{Bits 23--20} & 0\ 1\ 0\ 0 & \texttt{4} \\
\text{4} & \text{Bits 19--16} & 1\ 0\ 0\ 1 & \texttt{9} \\
\text{5} & \text{Bits 15--12} & 0\ 0\ 0\ 0 & \texttt{0} \\
\text{6} & \text{Bits 11--8} & 0\ 0\ 0\ 0 & \texttt{0} \\
\text{7} & \text{Bits 7--4} & 0\ 0\ 0\ 0 & \texttt{0} \\
\text{8} & \text{Bits 3--0} & 0\ 1\ 0\ 0 & \texttt{4} \\
\end{array}
\]

\textbf{Step 5: Write the Hexadecimal Representation}

Combining the hexadecimal digits from each group:
\[
\boxed{\texttt{0xAC490004}}
\]

