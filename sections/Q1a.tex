\textbf{Solution:}

\textbf{Step 1: Convert the Hex String to Binary}

First, convert each hex digit to its 4-bit binary equivalent:

\[
\begin{array}{ccccccccc}
\text{Hex:} & 2 & 0 & A & 3 & 0 & 0 & 1 & 2 \\
\text{Bin:} & 0010 & 0000 & 1010 & 0011 & 0000 & 0000 & 0001 & 0010 \\
\end{array}
\]

\textbf{Step 2: Identify the Opcode}

Opcode (bits 31-26): \(001000\). Which corresponds to the \texttt{ADDI} instruction in MIPS.  \cite{MIPS32}

\textbf{Step 3: Break Down the Binary Instruction}

MIPS instructions come in: R-type, I-type, and J-type. Based on the opcode, we can determine the format. The instruction format for I-type (which includes \texttt{ADDI}) is:

\begin{itemize}
    \item \textbf{Opcode (bits 31-26)}: \(001000\) (binary) \(= 8\) (decimal)
    \item \textbf{rs (bits 25-21)}: \(00101\) (binary) \(= \$5\) (register 5)
    \item \textbf{rt (bits 20-16)}: \(00011\) (binary) \(= \$3\) (register 3)
    \item \textbf{Immediate (bits 15-0)}: \(0000000000010010\) (binary) \(= 18\) (decimal)
\end{itemize}



\textbf{Step 4: Assemble the Instruction}

Now, we can construct the instruction:

\[
\boxed{
\texttt{addi \$3, \$5, 18}
}
\quad \text{or} \quad
\boxed{
\texttt{addi \$v1, \$a1, 18}
}
\]

